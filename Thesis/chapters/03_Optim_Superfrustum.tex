% !TeX root = ../main.tex
% Add the above to each chapter to make compiling the PDF easier in some editors.

\chapter{Stereo Rendering Optimization - Input reduction}
When looking at real-time rendering as it is done today - albeit from a strongly simplified perspective - the CPU could be described as an employer and the GPU as an employee. For each frame, the CPU produces certain render tasks and supplies the necessary information such as draw calls, shader parameters, buffers and so forth. The GPU then consumes these tasks and associated items and does the computational brunt to produce the required results. 
If one wants to speed up this process, there are two major ways. One is to reduce the amount of data put into the pipeline so less data needs to be processed overall, the other way is to increase the efficiency of the processing itself.  
This first chapter of optimization approaches presents ways of reducing the amount of data or work input. 
Note here that only a subset of the listed methods was implemented due to time constraints. 

\section{(Hierarchical) Frustum culling}
The following options build on top of the regular frustum culling concept. In this, the objects in the scene are checked against a camera frustum whether they are inside or outside or intersecting with the surface of the frustum. The checks themselves can be generally optimized in various ways, regardless of stereoscopy. Often, only an object's bounding geometry is checked. Collections of objects can be pre-computed so larger numbers may be discarded at once. An advanced option of culling is to delegate the calculations into a GPU compute shader so potentially less data needs to be transferred from the CPU per frame and much higher vector/matrix calculation is gained in exchange for slower branching. Some modern renderers also do very granular culling like bitmasked checks of precomputed triangle sets, as seen in Ubisoft's Anvil Next engine used in \textit{Assassin's Creed: Origins}\cite{Haar.2015}. 
Another optional layer of the culling process is to maintain hierarchical container structures for the scene objects so larger numbers can be discarded or included early on. 
For all those options the goal is the same: to obtain the list or buffer of objects visible by the given camera frustum in the scene. 

\section{Frustum (\& distance) culling in \textit{\gls{Tachyon}}}
The frustum culling approach in \textit{\gls{Tachyon}} was developed specifically for this thesis as a requirement for Superfrustum culling (\autoref{SFrust}) and as a sensible general optimization, is fully CPU-based and utilizes pre-computed hierarchical draw buffers. More specifically, at startup the scene is divided into a coarse grid of \codeword{chunk}s where each such \codeword{chunk} possesses an octree. Also at startup, a thread pool with the double the number of detected hardware threads is created. At asset load time, these octrees are populated with the loaded objects through optimistic size-aware insertion. Each cell of a tree then pre-computes a draw buffer containing a combined draw call for all objects associated with that cell. These buffers can be recomputed at any time, but the operation should be avoided at runtime as it incurs costly CPU to GPU transfers. 
In a culling pass, first all \codeword{chunk}s within a certain draw distance radius of the camera are chosen, so there is an additional very primitive distance culling taking place. Then each \codeword{chunk} submits a culling call using its tree to the thread pool. Each such call works as follows: \\

\begin{figure}[h]
  \centering
  \begin{tabular}{c}
  \begin{lstlisting}[language=C++]
pseudo SceneChunk::FrustumCull()
{
	out	draw-calls[camfrusta.size]; 
	for all leftover octree cells
	{
		for all camera frusta
		{
			if cell is valid
			var checkResult = frustumCheck(frustum, cell);
			if checkResult == INSIDE
			{
				add overall cell draw-call to draw-call list; 
			} else if checkResult == INTERSECT
			{
				add 8 child cells to leftover list; 
				add local cell draw-call to draw-call list; 
			}
		}
	}
	
	for all camera frusta
	{
		unique_lock(cullMutex); 
		sort and aggregate draw-call list; 
		add draw-call list to per-pipeline draw-call collection; 
	}
}
  \end{lstlisting}
  \end{tabular}
  \caption[SceneChunk's FrustumCull()]{per scene chunk frustum culling procedure (shortened pseudo code)}\label{fig:lst_SceneChunk_FrustumCull}
\end{figure}

In essence, the octree is looped through in a hierarchical fashion until either all remaining subcells of the current hierarchy level are outside of the frustum or no more subcells are left to check. Any resulting draw-calls are sorted to be memory layout friendly and aggregated into as few calls as possible to reduce invocation cost. \\

However, one problematic area remains and that is Z ordering. Modern graphics pipelines will perform early Z discard during the fragment stage. If a fragment fails the depth test for a given draw call, meaning its geometry would be occluded by triangles already written to the fragment, the shader will skip any further calculation for this geometry and fragment. For this to take hold in performance improvement, the draw calls need to be issued in a manner where geometry closer to the camera is drawn first. If draws are done in reverse order, the pipeline will draw distant geometry first and subsequent closer geometry at the same fragment will naturally not fail the depth test and overwrite the fragment. Effectively, all prior writes to any such fragment in that frame would be wasted and go unseen. This issue is called overdraw and can significantly deteriorate GPU render times in extreme cases. 
One such example would be the following, very dense synthetic test scene: 

\begin{figure}[htb]
  \centering
  \includegraphics[width=0.9\textwidth]{pictures/scene_chaos}
  \caption{Sample scene with high object density, far draw distance and high degree of overdraw when rendered without per-frame Z ordering, screenshot taken of \textit{\gls{Tachyon}}'s desktop viewport} \label{fig:scene_chaos}
\end{figure}

If the draw calls for each populated octree cell are issued back to front, the overdraw of many dozens if not hundreds of layers can push frametimes upward to the point where the GPU is geometry-bottlenecked in the depth test, while issuing the calls front to back may result in sufficiently low frametimes.
At this time, rtvklib does not employ such call reordering, meaning due to the nature of the octree cells' pre-recorded command buffers, overdraw cannot be avoided without a significant rewrite of these pre-recordings.

\section{Superfrustum Culling} \label{SFrust}
The basic idea behind so-called \textit{Superfrustum Culling} is to do regular single frustum culling despite rendering into two cameras, one per eye. The naive way of extending the frustum concept to a stereoscopic camera is to add a second frustum so there is one per eye, then perform the culling check for both frusta and merge the results. \\
As is easily visible from \textcolor{red}{[TODO: illustration of stereo frusta]}, the spatial proximity of of these two frusta leads to a large overlap volume, especially as field of view increases with more advanced headsets. One possible strategy to leverage more performance when culling two eyes is the Superfrustum, assuming the frustum is the common six sided trapezoid. Cass Everitt of Facebook Technologies LLC, formerly Oculus LLC, has suggested this approach and provided computation sketches on his social media back in 2015 \cite{Everitt.2015}(\autoref{fig:Everitt_Superfrustum}), and Nick Whiting at Oculus Connect 4 teased it as a future addition to Unreal Engine 4 \cite{Whiting.2017}. The idea is to combine the left and right eye frusta by taking the respective widest outer FOV tangent - usually the left eye's right side and the right eye's left side - and using these as the new side tangents of the superfrustum. Another way to express these is to take the widest half opening angles of each eye and adding them up to a combined opening angle. Similar is done for the top and bottom tangents, although these will usually be nearly identical for the two eyes. \\
A pitfall of the superfrustum is its necessary depth recession. This is easy to visualize when combining the two frusta by extending aforementioned side tangents backwards until they cross. The meeting point of this step is the new origin of the superfrustum, slightly recessed behind the two separate eyes. 
Vivien Oddou of Silicon Studios offered a generalized way to compute this recession for non-mirrored eye orientation \cite{Oddou.23.05.2017}(\autoref{fig:Oddou_Asymmetry}), while Everitt has extended his sketches by an asymmetry normalization\cite{Everitt.2015b}. Both of these are important to consider as virtual reality headsets can have slightly canted and asymmetrical lenses, either by design or by manufacturing tolerance. Ignoring these two corrections may still result in a sufficient superfrustum if computed conservatively but should be included for fully correct setups. 
While this superfrustum naturally eliminates all overlap of the naive variant, it in turn includes small false positive regions, notably the triangular void found close to the origin points between the two eye frusta and potential side edge regions in the case of asymmetrical lens orientations. In a typical application, the performance cost of these is negligible. \\

\begin{figure}[htb]
  \centering
  \includegraphics[width=0.7\textwidth]{pictures/Everitt_Superfrustum_Crop}
  \caption{Symmetric Superfrustum (cropped to geometric construction)\cite{Everitt.2015} \textcolor{red}{[TODO: footnote with source/courtesy]}} \label{fig:Everitt_Superfrustum}
\end{figure}

\begin{figure}[htb]
  \centering
  \includegraphics[width=0.6\textwidth]{pictures/Oddou_Asymmetry}
  \caption{Non-mirrored superfrustum recession\cite{Oddou.23.05.2017} \textcolor{red}{[TODO: footnote with source/courtesy]}} \label{fig:Oddou_Asymmetry}
\end{figure}

\subsection{Estimated impact}
The impact of using a Superfrustum will depend on the type of frustum culling math done and the combination with other techniques. \\
On its own, with a CPU based culling pass, only an appreciable benefit in CPU rendering time is to be expected, as the number of frustum checks will be reduced by up to 50\% and only a single buffer needs to be transferred to the graphics unit. The GPU itself still needs to render each eye separately, including all vertex transformations, pixel shading and so forth. The specific impact in the case of Tachyon is elaborated on in \autoref{results}. \\
Superfrustum culling when performed directly on the GPU obviously has great potential to significantly cut down on related compute work, once again to the effect of up to 50\% versus a baseline dual frustum culling. An interesting point to consider is whether any of the culling data needs to be synchronized back to the CPU, as, when not, the GPU compute workload will only depend on a single small buffer or pushconstant transfer containing the camera parameters. If, however, the resulting culling set is transferred back to the CPU, for example for preprocessing of the next frame, this transfer will present another speedup limiter. 

\subsection{Implementation specifics}
Facilitating Superfrustum Culling in Tachyon required only straight-forward changes. At creation time of the virtual camera, the superfrustum is computed from the given OpenVR eye parameters following Everitt and Oddou's way. 
Assuming the two eye projections are asymmetric, they need to be symmetricized. For this, after grabbing each eye's projection matrix from OpenVR for the desired near and far clip distances, the M[0][0] and M[0][2] values of each matrix are of interest. These two values represent the OpenGL clipping space's X coordinate's scalars. OpenVR through SteamVR uses the same coordinate layout. At a Z depth of -1, these two scalar terms are entirely dependent on center point and width of the projection cube. Solving them for center and width then gives $center = \frac{M[0][2]}{M[0][0]}$ and $width = \frac{2}{M[0][0]}$. 
A new symmetric projection then obviously could be constructed from this center and width by taking the center point and stepping "sideways" by half width. The new \textit{right}, or \textit{r} value, for example, will be $\textit{r}_{sym} = abs(center) + \frac{width}{2}$, which if substituted by the matrix dependent fractions comes out to $abs(\frac{M[0][2]}{M[0][0]}) + \frac{1}{M[0][0]}$. This can be solved for a new $M_{sym}[0][0] = \frac{M[0][0]}{abs(M[0][2])+1}$ and $M_{sym}[0][2] = 0$ (see \cite{Everitt.2015b}).
In the next step, these new M\textsubscript{sym}[0][0] values for both eyes are inserted into the recession math by Oddou\cite{Oddou.23.05.2017}. For non-mirrored and asymmetric eyes, a superfrustum recession is simply calculated as $\frac{ipd}{tan\theta _{0} + tan\theta _{1}}$ with $\theta \textsubscript{i}$ being the respective center-to-outside opening angle of each eye. Conveniently with the previously calculated M\textsubscript{sym} values, these tangents of angles are equal to the reciprocal of the respective M\textsubscript{sym}[0][0]. As such for the recession we get $\frac{ipd}{\frac{1}{M_{sym}[0][0]_{l}}+\frac{1}{M_{sym}[0][0]_{r}}}$. 
The calculated superfrustum recession and new combined field of view angles are saved and used every frame when re-transforming the superfrustum. The frustum transformation uses a simple geometric approach where the camera's world position, forward and up vectors in conjunction with the near and far distance, field of view and aspect ratio are extruded into the six planes of the frustum volume. 
The per-frame culling pass of \textit{\gls{Tachyon}} then naturally only checks against this single frustum and returns a single set of draw commands which are sent to both eyes. 

\section{Round Robin Culling}
Another culling variant specific to stereoscopy uses the round robin principle. Again, the concept springs from the desire to avoid frusta overlap but instead of combining the frusta, it exploits a common property of current stereoscopy rendering techniques. As modern headsets use circular lens optics and flat displays distorting the displayed image, the framebuffers commonly need to be warped to compensate so the picture looks undistorted to the user. As a result, a lot of the edge data of the image is either considerably pushed together or outside of the visible area of the HMD displays.
This conservative property means a virtual eye frustum can be smaller than the technical frustum of that respective eye and false negative discards in these edge regions would go unseen by the user.
Assuming both eyes of the headset have similar opening angles and parallel or nearly parallel viewing direction, the overlap of the two frusta would encompass the entire stereo-visible volume. It follows that only culling for one of the eye frusta would already give a sufficient representation of the actually visible scene. \\

There is still a possibility of missing a few edge cases with this alone. So the extension of the idea to actual round robin assumes another common property of modern \gls{VR} headsets, namely high refresh rates. Many of these aim for at least 80Hz (Rift S, \gls{WMR}) ranging up to 144Hz (Index experimental mode) image refresh to give the user a smooth visual sensation.
Exactly halving that refresh rate and reprojecting images for two refresh cycles with some pixel interpolation is an established way to still provide an acceptable experience with minor visual artifacting on slower devices as demonstrated by Oculus LLC's Asynchronous Space Warp\cite{Beeler.2016} and SteamVR's reprojection\cite{ValveCorporation.2018} features. Subsequently, a conceivable compromise is to alternate which frustum is used for culling in a round robin fashion so that even if edge cases include visible false negatives, they only persist for one frame at a time. In the worst case this would manifest as shimmering or flickering at the outer edges of the visible screen area. \\

Overall this makes Round Robin Culling a viable candidate on systems with very limited culling performance but the tight constraints for sufficiently accurate results make it unfit as a general recommendation. 

\begin{figure}[htb]
  \centering
  \includegraphics[width=0.9\textwidth]{pictures/placeholder}
  \caption{\textcolor{red}{[TODO: small illustration?]}} \label{fig:blob}
\end{figure}

\section{Conical Frustum Culling}
This third alternative culling extension targets the circular shape of HMD lenses for leverage. Coming back to the conservative framebuffer size from the previous section, the lenses lead to a lot of invisible area in the corners of the display. Jonathan Hale attempts to demonstrate the method in his thesis \cite{Hale.2018} as both a contribution to the graphics middleware \textit{Magnum}\cite{Vondrus.30.10.2019} and a UE4 extension at Vhite Rabbit\cite{VhiteRabbit.08.01.2020} - albeit with limited success. However, his proof of concept shows the validity of the method. The traditional six sided trapezoid frustum is replaced by a cone encompassing a volumetric projection of the view through each respective lens as visualized in \autoref{fig:hale_cone}. \\
Hale examined various types of cone intersection math, including AABB and bounding sphere object checks against a spatially transformed frustum cone and the same checks for spatially inverse-transformed objects against an origin oriented frustum cone. While his results showed cone culling performing worse than traditional frustum culling, he notes optimization was not fully refined and all calculation was done on the CPU. 
Another note by Hale cautions that depending on the used HMD, a cone frustum may prove \textit{less} accurate than a trapezoid, such as for the Oculus Rift CV1. As some headsets show not a fully circular image through their lenses but rather the entire distorted frame, it may not be possible to fit a cone frustum within the traditional frustum and instead that cone may actually exceed the traditional dimensions and thus cull fewer objects. 
For appropriate headsets and more geometry-bound GPUs, this method may provide a small relief if using even faster cone intersection math. 

\begin{figure}[htb]
  \centering
  \includegraphics[height=6cm]{pictures/hale_cone}
  \caption{Point-cone intersection illustration by Hale (\cite{Hale.2018}, p. 21)} \label{fig:hale_cone}
\end{figure}

\section{Merging approaches}
A convenient side effect of these three presented optimizations is that they can in part be merged. For example, it is possible to do Conical Round Robin Frustum Culling in an effort to slice away as much of the conservative invisible area as possible and reduce the list of drawable objects to an optimistic minimum. It is also possible to construct a Conical Superfrustum aiming to avoid the mentioned edge false positives, albeit only feasible if the display per eye is square-like to avoid adding new false positive volume on other sides.
