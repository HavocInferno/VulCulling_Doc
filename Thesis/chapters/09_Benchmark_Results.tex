% !TeX root = ../main.tex
% Add the above to each chapter to make compiling the PDF easier in some editors.

\chapter{Performance benchmark results} \label{results}
Testing each combination of \textit{ON} and \textit{OFF} states of the four optimizations logically yields 16 permutations. This chapter aims to show the metrics declared in \autoref{bmark_metrics} for each permutation and explore causes and implications for those permutations that would reasonably be employed in an application. 
To get a frame of reference for any potential performance improvement - or degradation - a baseline measurement is necessary, meaning a set of data points with all four optimizations set to \textit{OFF}. This baseline as measured on \textbf{WS-Big} looks as follows: \\

\textcolor{red}{[TODO: baseline plot (tikz?)]}

\section{Individual impact}
\textcolor{red}{[TODO]}
\section{Combined and partially combined impact}
\textcolor{red}{[TODO]}