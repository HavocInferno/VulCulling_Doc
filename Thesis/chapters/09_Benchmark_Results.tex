% !TeX root = ../main.tex
% Add the above to each chapter to make compiling the PDF easier in some editors.

\chapter{Performance benchmark results} \label{results}
Testing each combination of \textit{ON} and \textit{OFF} states of the four optimizations logically yields 16 permutations. This chapter aims to show the metrics declared in \autoref{bmark_metrics} for each permutation and explore causes and implications for those permutations that would reasonably be employed in an application. 
To get a frame of reference for any potential performance improvement - or degradation - a baseline measurement is necessary, meaning a set of data points with all four optimizations set to \textit{OFF}. This baseline as measured on \textbf{WS-Big} looks as follows (\autoref{fig:bmark_baseline}): 

\begin{figure}[h!]
  \begin{center}
    \begin{tikzpicture}
      \begin{axis}[
          width=\linewidth, % Scale the plot to \linewidth
          height=0.5\linewidth,
          grid=major, 
          grid style={dashed,gray!30},
          %legend style={at={(0.5,-0.2)},anchor=north},
          legend cell align={left},
          legend columns=3,
          legend style={at={(1,1.11)},anchor=north east},
          xmin=0,
          xmax=5400,
          xlabel=Frame Index, % Set the labels
          x tick label style={rotate=90,anchor=east},
          ylabel=time in milliseconds,
          ylabel near ticks,
          scaled y ticks={real:1000},
          ytick scale label code/.code={}
        ]
        \addplot 
        % add a plot from table; you select the columns by using the actual name in
        % the .csv file (on top)
        table[x=frameindex,y=frametime,col sep=comma] {plots/MVIEW_OFF-SFRUST_OFF-SMASK_OFF-MFFR_OFF-VRBench-AVG.csv}; 
        \addlegendentry{Whole frame}
        
        \addplot 
        % add a plot from table; you select the columns by using the actual name in
        % the .csv file (on top)
        table[x=frameindex,y=culltime,col sep=comma] {plots/MVIEW_OFF-SFRUST_OFF-SMASK_OFF-MFFR_OFF-VRBench-AVG.csv}; 
        \addlegendentry{Culling only}
        
        \addplot 
        % add a plot from table; you select the columns by using the actual name in
        % the .csv file (on top)
        table[x=frameindex,y=gputime,col sep=comma] {plots/MVIEW_OFF-SFRUST_OFF-SMASK_OFF-MFFR_OFF-VRBench-AVG.csv}; 
        \addlegendentry{GPU only}
      \end{axis}
    \end{tikzpicture}
    \caption{Baseline performance (all OFF, 10 run average)} \label{fig:bmark_baseline}
  \end{center}
\end{figure}

While these performance numbers were recorded, the sensors of the RTX 2080 and i7 6700 were monitored using the tool \textit{HWiNFO64}. For the duration of the test, this tool attests utilization averages of 80\% for the GPU core, 37\% memory capacity, 32\% for the GPU memory controller and 20\% for the PCIe bus link. 
For the CPU an overall average utilization of 49\% with each of the 8 threads averaging between 30-40\% was seen. RAM usage from asset load at program start up until termination reported 5.4GB for the application by itself. 

\section{Individual impact}
\textcolor{red}{[TODO]}

\begin{figure}[h!]
  \begin{center}
    \begin{tikzpicture}
      \begin{axis}[
          width=\linewidth, % Scale the plot to \linewidth
          height=0.5\linewidth,
          grid=major, 
          grid style={dashed,gray!30},
          %legend style={at={(0.5,-0.2)},anchor=north},
          legend cell align={left},
          legend columns=3,
          legend style={at={(1,1.11)},anchor=north east},
          xmin=0,
          xmax=5400,
          xlabel=Frame Index, % Set the labels
          x tick label style={rotate=90,anchor=east},
          ylabel=time in milliseconds,
          ylabel near ticks,
          scaled y ticks={real:1000},
          ytick scale label code/.code={}
        ]
        \addplot 
        % add a plot from table; you select the columns by using the actual name in
        % the .csv file (on top)
        table[x=frameindex,y=frametime,col sep=comma] {plots/MVIEW_OFF-SFRUST_OFF-SMASK_OFF-MFFR_ON-VRBench-AVG.csv}; 
        \addlegendentry{Whole frame}
        
        \addplot 
        % add a plot from table; you select the columns by using the actual name in
        % the .csv file (on top)
        table[x=frameindex,y=culltime,col sep=comma] {plots/MVIEW_OFF-SFRUST_OFF-SMASK_OFF-MFFR_ON-VRBench-AVG.csv}; 
        \addlegendentry{Culling only}
        
        \addplot 
        % add a plot from table; you select the columns by using the actual name in
        % the .csv file (on top)
        table[x=frameindex,y=gputime,col sep=comma] {plots/MVIEW_OFF-SFRUST_OFF-SMASK_OFF-MFFR_ON-VRBench-AVG.csv}; 
        \addlegendentry{GPU only}
      \end{axis}
    \end{tikzpicture}
    \caption{MFFR ON performance (10 run average)} \label{fig:bmark_MFFR-ON}
  \end{center}
\end{figure}

\begin{figure}[h!]
  \begin{center}
    \begin{tikzpicture}
      \begin{axis}[
          width=\linewidth, % Scale the plot to \linewidth
          height=0.5\linewidth,
          grid=major, 
          grid style={dashed,gray!30},
          %legend style={at={(0.5,-0.2)},anchor=north},
          legend cell align={left},
          legend columns=3,
          legend style={at={(1,1.11)},anchor=north east},
          xmin=0,
          xmax=5400,
          xlabel=Frame Index, % Set the labels
          x tick label style={rotate=90,anchor=east},
          ylabel=time in milliseconds,
          ylabel near ticks,
          scaled y ticks={real:1000},
          ytick scale label code/.code={}
        ]
        \addplot 
        % add a plot from table; you select the columns by using the actual name in
        % the .csv file (on top)
        table[x=frameindex,y=frametime,col sep=comma] {plots/MVIEW_OFF-SFRUST_OFF-SMASK_ON-MFFR_OFF-VRBench-AVG.csv}; 
        \addlegendentry{Whole frame}
        
        \addplot 
        % add a plot from table; you select the columns by using the actual name in
        % the .csv file (on top)
        table[x=frameindex,y=culltime,col sep=comma] {plots/MVIEW_OFF-SFRUST_OFF-SMASK_ON-MFFR_OFF-VRBench-AVG.csv}; 
        \addlegendentry{Culling only}
        
        \addplot 
        % add a plot from table; you select the columns by using the actual name in
        % the .csv file (on top)
        table[x=frameindex,y=gputime,col sep=comma] {plots/MVIEW_OFF-SFRUST_OFF-SMASK_ON-MFFR_OFF-VRBench-AVG.csv}; 
        \addlegendentry{GPU only}
      \end{axis}
    \end{tikzpicture}
    \caption{SMASK ON performance (10 run average)} \label{fig:bmark_SMASK-ON}
  \end{center}
\end{figure}

\section{Combined and partially combined impact}
\textcolor{red}{[TODO]}