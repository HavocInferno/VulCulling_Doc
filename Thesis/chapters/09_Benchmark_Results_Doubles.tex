% !TeX root = ../main.tex
% Add the above to each chapter to make compiling the PDF easier in some editors.

\section{Combined impact}
Next, the possible combination of multiple optimizations is examined. Similar to the individual benchmarks, Stencil Masking tightens framepacing by a small margin (\autoref{fig:bmark_SMASK_SFRUST} \& \autoref{fig:bmark_SMASK_MVIEW}) while Superfrustum use cancels out its CPU gains. Unless, that is, the Superfrustum is combined with Multiview rendering in \autoref{fig:bmark_SFRUST_MVIEW}, in which case it results in a minor reduction in cull time barely beating Multiview's own multi-frustum optimization. The two configurations of Stereo Multiview with Stencil Masking and either Superfrustum \textit{ON} or \textit{OFF} (\autoref{fig:bmark_SMASK_MVIEW} \& \autoref{fig:bmark_SMASK_SFRUST_MVIEW}) showcase the greatest promise in overall frame time and GPU render time. 

\begin{figure}[H]
  \begin{center}
    \begin{tikzpicture}
      \begin{axis}[
          width=\linewidth, % Scale the plot to \linewidth
          height=0.5\linewidth,
          grid=major, 
          grid style={dashed,gray!30},
          %legend style={at={(0.5,-0.2)},anchor=north},
          legend cell align={left},
          legend columns=3,
          legend style={at={(1,1.11)},anchor=north east},
          xmin=0,
          xmax=5400,
          xlabel=Frame Index, % Set the labels
          x tick label style={rotate=90,anchor=east},
          ymin=0,
          ymax=16000,
          ylabel=time in milliseconds,
          ylabel near ticks,
          scaled y ticks={real:1000},
          ytick scale label code/.code={}
        ]
        \addplot 
        % add a plot from table; you select the columns by using the actual name in
        % the .csv file (on top)
        table[x=frameindex,y=frametime,col sep=comma] {plots/WS-Big/MVIEW_OFF-SFRUST_ON-SMASK_ON-MFFR_OFF-VRBench-AVG.csv}; 
        \addlegendentry{Whole frame}
                
        \addplot 
        % add a plot from table; you select the columns by using the actual name in
        % the .csv file (on top)
        table[x=frameindex,y=gputime,col sep=comma] {plots/WS-Big/MVIEW_OFF-SFRUST_ON-SMASK_ON-MFFR_OFF-VRBench-AVG.csv}; 
        \addlegendentry{GPU only}
        
        \addplot 
        % add a plot from table; you select the columns by using the actual name in
        % the .csv file (on top)
        table[x=frameindex,y=culltime,col sep=comma] {plots/WS-Big/MVIEW_OFF-SFRUST_ON-SMASK_ON-MFFR_OFF-VRBench-AVG.csv}; 
        \addlegendentry{Culling only}
      \end{axis}
    \end{tikzpicture}
    \caption{SMASK \& SFRUST ON performance (10 run average)} \label{fig:bmark_SMASK_SFRUST}
  \end{center}
\end{figure}

\begin{figure}[H]
  \begin{center}
    \begin{tikzpicture}
      \begin{axis}[
          width=\linewidth, % Scale the plot to \linewidth
          height=0.5\linewidth,
          grid=major, 
          grid style={dashed,gray!30},
          %legend style={at={(0.5,-0.2)},anchor=north},
          legend cell align={left},
          legend columns=3,
          legend style={at={(1,1.11)},anchor=north east},
          xmin=0,
          xmax=5400,
          xlabel=Frame Index, % Set the labels
          x tick label style={rotate=90,anchor=east},
          ymin=0,
          ymax=16000,
          ylabel=time in milliseconds,
          ylabel near ticks,
          scaled y ticks={real:1000},
          ytick scale label code/.code={}
        ]
        \addplot 
        % add a plot from table; you select the columns by using the actual name in
        % the .csv file (on top)
        table[x=frameindex,y=frametime,col sep=comma] {plots/WS-Big/MVIEW_ON-SFRUST_OFF-SMASK_ON-MFFR_OFF-VRBench-AVG.csv}; 
        \addlegendentry{Whole frame}
                
        \addplot 
        % add a plot from table; you select the columns by using the actual name in
        % the .csv file (on top)
        table[x=frameindex,y=gputime,col sep=comma] {plots/WS-Big/MVIEW_ON-SFRUST_OFF-SMASK_ON-MFFR_OFF-VRBench-AVG.csv}; 
        \addlegendentry{GPU only}
        
        \addplot 
        % add a plot from table; you select the columns by using the actual name in
        % the .csv file (on top)
        table[x=frameindex,y=culltime,col sep=comma] {plots/WS-Big/MVIEW_ON-SFRUST_OFF-SMASK_ON-MFFR_OFF-VRBench-AVG.csv}; 
        \addlegendentry{Culling only}
      \end{axis}
    \end{tikzpicture}
    \caption{SMASK \& MVIEW ON performance (10 run average)} \label{fig:bmark_SMASK_MVIEW}
  \end{center}
\end{figure}

\begin{figure}[H]
  \begin{center}
    \begin{tikzpicture}
      \begin{axis}[
          width=\linewidth, % Scale the plot to \linewidth
          height=0.5\linewidth,
          grid=major, 
          grid style={dashed,gray!30},
          %legend style={at={(0.5,-0.2)},anchor=north},
          legend cell align={left},
          legend columns=3,
          legend style={at={(1,1.11)},anchor=north east},
          xmin=0,
          xmax=5400,
          xlabel=Frame Index, % Set the labels
          x tick label style={rotate=90,anchor=east},
          ymin=0,
          ymax=16000,
          ylabel=time in milliseconds,
          ylabel near ticks,
          scaled y ticks={real:1000},
          ytick scale label code/.code={}
        ]
        \addplot 
        % add a plot from table; you select the columns by using the actual name in
        % the .csv file (on top)
        table[x=frameindex,y=frametime,col sep=comma] {plots/WS-Big/MVIEW_ON-SFRUST_ON-SMASK_OFF-MFFR_OFF-VRBench-AVG.csv}; 
        \addlegendentry{Whole frame}
                
        \addplot 
        % add a plot from table; you select the columns by using the actual name in
        % the .csv file (on top)
        table[x=frameindex,y=gputime,col sep=comma] {plots/WS-Big/MVIEW_ON-SFRUST_ON-SMASK_OFF-MFFR_OFF-VRBench-AVG.csv}; 
        \addlegendentry{GPU only}
        
        \addplot 
        % add a plot from table; you select the columns by using the actual name in
        % the .csv file (on top)
        table[x=frameindex,y=culltime,col sep=comma] {plots/WS-Big/MVIEW_ON-SFRUST_ON-SMASK_OFF-MFFR_OFF-VRBench-AVG.csv}; 
        \addlegendentry{Culling only}
      \end{axis}
    \end{tikzpicture}
    \caption{SFRUST \& MVIEW ON performance (10 run average)} \label{fig:bmark_SFRUST_MVIEW}
  \end{center}
\end{figure}

\begin{figure}[H]
  \begin{center}
    \begin{tikzpicture}
      \begin{axis}[
          width=\linewidth, % Scale the plot to \linewidth
          height=0.5\linewidth,
          grid=major, 
          grid style={dashed,gray!30},
          %legend style={at={(0.5,-0.2)},anchor=north},
          legend cell align={left},
          legend columns=3,
          legend style={at={(1,1.11)},anchor=north east},
          xmin=0,
          xmax=5400,
          xlabel=Frame Index, % Set the labels
          x tick label style={rotate=90,anchor=east},
          ymin=0,
          ymax=16000,
          ylabel=time in milliseconds,
          ylabel near ticks,
          scaled y ticks={real:1000},
          ytick scale label code/.code={}
        ]
        \addplot 
        % add a plot from table; you select the columns by using the actual name in
        % the .csv file (on top)
        table[x=frameindex,y=frametime,col sep=comma] {plots/WS-Big/MVIEW_ON-SFRUST_ON-SMASK_ON-MFFR_OFF-VRBench-AVG.csv}; 
        \addlegendentry{Whole frame}
                
        \addplot 
        % add a plot from table; you select the columns by using the actual name in
        % the .csv file (on top)
        table[x=frameindex,y=gputime,col sep=comma] {plots/WS-Big/MVIEW_ON-SFRUST_ON-SMASK_ON-MFFR_OFF-VRBench-AVG.csv}; 
        \addlegendentry{GPU only}
        
        \addplot 
        % add a plot from table; you select the columns by using the actual name in
        % the .csv file (on top)
        table[x=frameindex,y=culltime,col sep=comma] {plots/WS-Big/MVIEW_ON-SFRUST_ON-SMASK_ON-MFFR_OFF-VRBench-AVG.csv}; 
        \addlegendentry{Culling only}
      \end{axis}
    \end{tikzpicture}
    \caption{SMASK \& SFRUST \& MVIEW ON performance (10 run average)} \label{fig:bmark_SMASK_SFRUST_MVIEW}
  \end{center}
\end{figure}

\subsection{Comparison of combined optimizations}
To get an even better understanding of how each of these four configurations stack up against each other and against the baseline, another set of comparison plots is examined. 
Due to the established negative impact on GPU times of Superfrustum use without Multiview rendering, the combination of Stencil Masking and Superfrustum just barely outperforms the baseline overall (\autoref{fig:bmark_combined_whole}), by a lower margin (median 4.3\% and avg 3.8\% vs 4.5\% and 4.3\%) than Stencil Masking on its own. Any combination involving Stereo Multiview leads to significant improvement as expected. Interestingly, however, combining Multiview with the Superfrustum performs worse compared to just Multiview on its own by a small margin (\autoref{fig:bmark_MVIEW_vs_MVIEWSFRUST}). \\
When it comes to best overall performance, the combinations of Multiview with Stencil Masking and of Multiview with Stencil Masking and Superfrustum are close together, but the former surprisingly comes out ahead by a small margin. It appears the small amount of blind volume of the Superfrustum compared to the separata frusta affords a small advantage in cull time but trades it for a minor increase in GPU rendering load. 
CPU time benchmarks in \autoref{fig:bmark_combined_CPU} mirror the results already seen for the individual impacts with Superfrustum improving on the baseline by a noticeable margin and any Multiview configuration extending this lead even more. Cull time also shows a similar story to \autoref{singles_comp} with Superfrustum and Multiview configurations being effectively on par, underlined by the near-identical cull time medians/averages in \autoref{tab:combined_medians}. The combination of Superfrustum with Multiview yields another miniscule cull time improvement but once more trades it for increased GPU rendering load as seen in \autoref{fig:bmark_combined_GPU}. \\
As such, the configuration of Stereo Multiview \textit{ON}, Stencil Masking \textit{ON}, Superfrustum \textit{OFF} and \gls{MFFR} \textit{OFF} performs the best on the used machine with a median frametime reduction of 23.7\% and average reduction of 24.4\%. Expressed inversely as framerate, this equals a boost from 88.57fps median and 85.11fps average to 116.14fps (+31.1\%) and 112.61fps (+32.3\%) respectively. \\

Considering all these results, the most viable optimizations to employ are as follows: 
\begin{itemize}
\item Stencil Masking (SMASK) lends itself as a quickly implemented and lightweight way of gaining a moderate pixel shader performance boost, only requiring a single 8 bit framebuffer layer per eye at most
\item Stereo Multiview (MVIEW), while reliant on GPU and API support and thus only available on more recent architectures - and only accelerated in hardware on very modern chipsets such as the RTX 2080 used here - , can yield significant gains with little to no other significant tradeoff
\item Stereo Multiview and Stencil Masking naturally promises to be a synergetic combination of approaches with no major downside
\item Stereo Multiview and Superfrustum (SFRUST) Culling can be combined for good results, but may only scale favorably if the target system is constrained in culling performance so the savings afforded by the Superfrustum outweigh the slight increase in GPU load
\item Combining all three of these optimizations is subject to the same restriction as Multiview plus Superfrustum but the addition of Stencil Masking in this case can regain if not overcome the Superfrustum GPU performance loss
\end{itemize}
Other optimization combinations presented up to this point are either non-sensical such as Superfrustum Culling without Multiview acceleration which results in too high a loss in GPU performance or are functionally defective in the case of \gls{MFFR} combinations in the state implemented for this thesis. However, further work and improvement on the latter is expected to return positive gains given adequate virtual environments. 

\begin{table}[H]
  \caption[Median/avg timings for frame/gpu/cull for baseline and combined optimizations]{Median and average (arithmetic mean) timings for frametime, gpu-time and cull-time for baseline and combined optimizations (rounded to two decimal places)}\label{tab:combined_medians}
  \centering
  \begin{tabular}{l || S S | S S | S S}
    \toprule
  	\multirow{2}{*}{Config} & 
  		\multicolumn{2}{c}{frame} & 
  		\multicolumn{2}{c}{gpu} & 
  		\multicolumn{2}{c}{cull} \\
       & {median} & {avg} & {median} & {avg} & {median} & {avg} \\
    \midrule
      baseline 	& 11.29 & 11.75 & 9.06 & 9.36 & 1.97 & 2.04 \\
      SMASK \& SFRUST 	& 10.80 & 11.30 & 8.93 & 9.25 & 1.66 & 1.72 \\
      SMASK \& MVIEW 	& 8.61 & 8.88 & 6.84 & 6.95 & 1.65 & 1.70 \\
      SFRUST \& MVIEW	& 9.30 & 9.62 & 7.54 & 7.72 & 1.60 & 1.66 \\
      SMASK \& SFRUST \& MVIEW 	& 8.80 & 9.07 & 7.05 & 7.15 & 1.62 & 1.69 \\
    \bottomrule
  \end{tabular}
\end{table}

\begin{figure}[H]
  \begin{center}
    \begin{tikzpicture}
      \begin{axis}[
          width=\linewidth, % Scale the plot to \linewidth
          height=0.5\linewidth,
          grid=major, 
          grid style={dashed,gray!30},
          %legend style={at={(0.5,-0.2)},anchor=north},
          legend cell align={left},
          legend columns=3,
          legend style={at={(1,1.11)},anchor=north east},
          xmin=0,
          xmax=5400,
          xlabel=Frame Index, % Set the labels
          x tick label style={rotate=90,anchor=east},
          ymin=4000,
          ymax=16000,
          ylabel=time in milliseconds,
          ylabel near ticks,
          scaled y ticks={real:1000},
          ytick scale label code/.code={}
        ]
        \addplot 
        % add a plot from table; you select the columns by using the actual name in
        % the .csv file (on top)
        table[x=frameindex,y=frametime,col sep=comma] {plots/WS-Big/MVIEW_OFF-SFRUST_OFF-SMASK_OFF-MFFR_OFF-VRBench-AVG.csv}; 
        \addlegendentry{all OFF}
        
        \addplot 
        % add a plot from table; you select the columns by using the actual name in
        % the .csv file (on top)
        table[x=frameindex,y=frametime,col sep=comma] {plots/WS-Big/MVIEW_OFF-SFRUST_ON-SMASK_ON-MFFR_OFF-VRBench-AVG.csv}; 
        \addlegendentry{SMASK \& SFRUST}
        
        \addplot 
        % add a plot from table; you select the columns by using the actual name in
        % the .csv file (on top)
        table[x=frameindex,y=frametime,col sep=comma] {plots/WS-Big/MVIEW_ON-SFRUST_OFF-SMASK_ON-MFFR_OFF-VRBench-AVG.csv}; 
        \addlegendentry{SMASK \& MVIEW}
        
        \addplot 
        % add a plot from table; you select the columns by using the actual name in
        % the .csv file (on top)
        table[x=frameindex,y=frametime,col sep=comma] {plots/WS-Big/MVIEW_ON-SFRUST_ON-SMASK_OFF-MFFR_OFF-VRBench-AVG.csv}; 
        \addlegendentry{SFRUST \& MVIEW}
        
        \addplot 
        % add a plot from table; you select the columns by using the actual name in
        % the .csv file (on top)
        table[x=frameindex,y=frametime,col sep=comma] {plots/WS-Big/MVIEW_ON-SFRUST_ON-SMASK_ON-MFFR_OFF-VRBench-AVG.csv}; 
        \addlegendentry{SMASK \& SFRUST \& MVIEW}
      \end{axis}
    \end{tikzpicture}
    \caption{comparison (Whole frame time)} \label{fig:bmark_combined_whole}
  \end{center}
\end{figure}

\begin{figure}[H]
  \begin{center}
    \begin{tikzpicture}
      \begin{axis}[
          width=\linewidth, % Scale the plot to \linewidth
          height=0.5\linewidth,
          grid=major, 
          grid style={dashed,gray!30},
          %legend style={at={(0.5,-0.2)},anchor=north},
          legend cell align={left},
          legend columns=3,
          legend style={at={(1,1.11)},anchor=north east},
          xmin=0,
          xmax=5400,
          xlabel=Frame Index, % Set the labels
          x tick label style={rotate=90,anchor=east},
          ymin=1000,
          ymax=4000,
          ylabel=time in milliseconds,
          ylabel near ticks,
          scaled y ticks={real:1000},
          ytick scale label code/.code={}
        ]
        \addplot 
        % add a plot from table; you select the columns by using the actual name in
        % the .csv file (on top)
        table[x=frameindex,y=cputime,col sep=comma] {plots/WS-Big/MVIEW_OFF-SFRUST_OFF-SMASK_OFF-MFFR_OFF-VRBench-AVG.csv}; 
        \addlegendentry{all OFF}
        
        \addplot 
        % add a plot from table; you select the columns by using the actual name in
        % the .csv file (on top)
        table[x=frameindex,y=cputime,col sep=comma] {plots/WS-Big/MVIEW_OFF-SFRUST_ON-SMASK_ON-MFFR_OFF-VRBench-AVG.csv}; 
        \addlegendentry{SMASK \& SFRUST}
        
        \addplot 
        % add a plot from table; you select the columns by using the actual name in
        % the .csv file (on top)
        table[x=frameindex,y=cputime,col sep=comma] {plots/WS-Big/MVIEW_ON-SFRUST_OFF-SMASK_ON-MFFR_OFF-VRBench-AVG.csv}; 
        \addlegendentry{SMASK \& MVIEW}
        
        \addplot 
        % add a plot from table; you select the columns by using the actual name in
        % the .csv file (on top)
        table[x=frameindex,y=cputime,col sep=comma] {plots/WS-Big/MVIEW_ON-SFRUST_ON-SMASK_OFF-MFFR_OFF-VRBench-AVG.csv}; 
        \addlegendentry{SFRUST \& MVIEW}
        
        \addplot 
        % add a plot from table; you select the columns by using the actual name in
        % the .csv file (on top)
        table[x=frameindex,y=cputime,col sep=comma] {plots/WS-Big/MVIEW_ON-SFRUST_ON-SMASK_ON-MFFR_OFF-VRBench-AVG.csv}; 
        \addlegendentry{SMASK \& SFRUST \& MVIEW}
      \end{axis}
    \end{tikzpicture}
    \caption{comparison (CPU time)} \label{fig:bmark_combined_CPU}
  \end{center}
\end{figure}

\begin{figure}[H]
  \begin{center}
    \begin{tikzpicture}
      \begin{axis}[
          width=\linewidth, % Scale the plot to \linewidth
          height=0.5\linewidth,
          grid=major, 
          grid style={dashed,gray!30},
          %legend style={at={(0.5,-0.2)},anchor=north},
          legend cell align={left},
          legend columns=3,
          legend style={at={(1,1.11)},anchor=north east},
          xmin=0,
          xmax=5400,
          xlabel=Frame Index, % Set the labels
          x tick label style={rotate=90,anchor=east},
          ymin=500,
          ymax=3500,
          ylabel=time in milliseconds,
          ylabel near ticks,
          scaled y ticks={real:1000},
          ytick scale label code/.code={}
        ]
        \addplot 
        % add a plot from table; you select the columns by using the actual name in
        % the .csv file (on top)
        table[x=frameindex,y=culltime,col sep=comma] {plots/WS-Big/MVIEW_OFF-SFRUST_OFF-SMASK_OFF-MFFR_OFF-VRBench-AVG.csv}; 
        \addlegendentry{all OFF}
        
        \addplot 
        % add a plot from table; you select the columns by using the actual name in
        % the .csv file (on top)
        table[x=frameindex,y=culltime,col sep=comma] {plots/WS-Big/MVIEW_OFF-SFRUST_ON-SMASK_ON-MFFR_OFF-VRBench-AVG.csv}; 
        \addlegendentry{SMASK \& SFRUST}
        
        \addplot 
        % add a plot from table; you select the columns by using the actual name in
        % the .csv file (on top)
        table[x=frameindex,y=culltime,col sep=comma] {plots/WS-Big/MVIEW_ON-SFRUST_OFF-SMASK_ON-MFFR_OFF-VRBench-AVG.csv}; 
        \addlegendentry{SMASK \& MVIEW}
        
        \addplot 
        % add a plot from table; you select the columns by using the actual name in
        % the .csv file (on top)
        table[x=frameindex,y=culltime,col sep=comma] {plots/WS-Big/MVIEW_ON-SFRUST_ON-SMASK_OFF-MFFR_OFF-VRBench-AVG.csv}; 
        \addlegendentry{SFRUST \& MVIEW}
        
        \addplot 
        % add a plot from table; you select the columns by using the actual name in
        % the .csv file (on top)
        table[x=frameindex,y=culltime,col sep=comma] {plots/WS-Big/MVIEW_ON-SFRUST_ON-SMASK_ON-MFFR_OFF-VRBench-AVG.csv}; 
        \addlegendentry{SMASK \& SFRUST \& MVIEW}
      \end{axis}
    \end{tikzpicture}
    \caption{comparison (Cull time)} \label{fig:bmark_combined_cull}
  \end{center}
\end{figure}

\begin{figure}[H]
  \begin{center}
    \begin{tikzpicture}
      \begin{axis}[
          width=\linewidth, % Scale the plot to \linewidth
          height=0.5\linewidth,
          grid=major, 
          grid style={dashed,gray!30},
          %legend style={at={(0.5,-0.2)},anchor=north},
          legend cell align={left},
          legend columns=3,
          legend style={at={(1,1.11)},anchor=north east},
          xmin=0,
          xmax=5400,
          xlabel=Frame Index, % Set the labels
          x tick label style={rotate=90,anchor=east},
          ymin=4000,
          ymax=14000,
          ylabel=time in milliseconds,
          ylabel near ticks,
          scaled y ticks={real:1000},
          ytick scale label code/.code={}
        ]
        \addplot 
        % add a plot from table; you select the columns by using the actual name in
        % the .csv file (on top)
        table[x=frameindex,y=gputime,col sep=comma] {plots/WS-Big/MVIEW_OFF-SFRUST_OFF-SMASK_OFF-MFFR_OFF-VRBench-AVG.csv}; 
        \addlegendentry{all OFF}
        
        \addplot 
        % add a plot from table; you select the columns by using the actual name in
        % the .csv file (on top)
        table[x=frameindex,y=gputime,col sep=comma] {plots/WS-Big/MVIEW_OFF-SFRUST_ON-SMASK_ON-MFFR_OFF-VRBench-AVG.csv}; 
        \addlegendentry{SMASK \& SFRUST}
        
        \addplot 
        % add a plot from table; you select the columns by using the actual name in
        % the .csv file (on top)
        table[x=frameindex,y=gputime,col sep=comma] {plots/WS-Big/MVIEW_ON-SFRUST_OFF-SMASK_ON-MFFR_OFF-VRBench-AVG.csv}; 
        \addlegendentry{SMASK \& MVIEW}
        
        \addplot 
        % add a plot from table; you select the columns by using the actual name in
        % the .csv file (on top)
        table[x=frameindex,y=gputime,col sep=comma] {plots/WS-Big/MVIEW_ON-SFRUST_ON-SMASK_OFF-MFFR_OFF-VRBench-AVG.csv}; 
        \addlegendentry{SFRUST \& MVIEW}
        
        \addplot 
        % add a plot from table; you select the columns by using the actual name in
        % the .csv file (on top)
        table[x=frameindex,y=gputime,col sep=comma] {plots/WS-Big/MVIEW_ON-SFRUST_ON-SMASK_ON-MFFR_OFF-VRBench-AVG.csv}; 
        \addlegendentry{SMASK \& SFRUST \& MVIEW}
      \end{axis}
    \end{tikzpicture}
    \caption{comparison (GPU time)} \label{fig:bmark_combined_GPU}
  \end{center}
\end{figure}

\begin{figure}[H]
  \begin{center}
    \begin{tikzpicture}
      \begin{axis}[
          width=\linewidth, % Scale the plot to \linewidth
          height=0.5\linewidth,
          grid=major, 
          grid style={dashed,gray!30},
          %legend style={at={(0.5,-0.2)},anchor=north},
          legend cell align={left},
          legend columns=3,
          legend style={at={(1,1.11)},anchor=north east},
          xmin=0,
          xmax=5400,
          xlabel=Frame Index, % Set the labels
          x tick label style={rotate=90,anchor=east},
          ymin=7000,
          ymax=13000,
          ylabel=time in milliseconds,
          ylabel near ticks,
          scaled y ticks={real:1000},
          ytick scale label code/.code={}
        ]
        \addplot 
        % add a plot from table; you select the columns by using the actual name in
        % the .csv file (on top)
        table[x=frameindex,y=frametime,col sep=comma] {plots/WS-Big/MVIEW_ON-SFRUST_OFF-SMASK_OFF-MFFR_OFF-VRBench-AVG.csv}; 
        \addlegendentry{MVIEW only}
        
        \addplot 
        % add a plot from table; you select the columns by using the actual name in
        % the .csv file (on top)
        table[x=frameindex,y=frametime,col sep=comma] {plots/WS-Big/MVIEW_ON-SFRUST_ON-SMASK_OFF-MFFR_OFF-VRBench-AVG.csv}; 
        \addlegendentry{SFRUST \& MVIEW}
      \end{axis}
    \end{tikzpicture}
    \caption{MVIEW only vs SFRUST \& MVIEW (Whole frame time)} \label{fig:bmark_MVIEW_vs_MVIEWSFRUST}
  \end{center}
\end{figure}