% !TeX root = ../main.tex
% Add the above to each chapter to make compiling the PDF easier in some editors.

\chapter{Outlook}
To recapitulate this thesis, the goal at the outset was to gather a collection of optimization methods specifically tailored for \gls{VR}. Of these methods, a subset was to be implemented in an industrial real-time visualization rendering engine. Finally these implementations were to be benchmarked in a high stress scenario to asses the performance of each optimization by itself and in conjunction with the remaining methods. This goal sprung from the hope to aggregate useful information and tangible numbers about ways to speed up \gls{VR} rendering by a significant margin. \\
While the list of presented optimization approaches is not exhaustive or complete, as new or more advanced methods regularly show up in this field and including every single way and variation simply exceeds the scope of this thesis, the provided chapters do in fact contain a plethora of information and elaboration on key avenues. These cover multiple angles such as GPU versus CPU performance gains, pipeline speedups and varying hardware architectures and which of their parts are involved. 
The implemented optimizations, while promising, did not all pan out as expected. Stencil Masking and Multiview Rendering show clear and tangible improvements in frametimes and are considered a success. Superfrustum Culling provides more of a tradeoff to alleviate stress and provide headroom on the CPU in exchange for higher GPU workloads and requires fitting circumstances to pay off in a target scenario. Monoscopic Far-Field Rendering finally may be among the interesting of the concepts presented, but in the practical implementation delivered incorrect and disappointing results. \\
Nonetheless, valuable insight for deployment of these four optimizations was gained and even \gls{MFFR} still shows promise given additional iteration and care to iron out the observed issues. It is our hope that the presented approaches and demonstrated results are of similar value to other efforts in the field. After all, every millisecond shaved off is precious in real-time graphics. 