\chapter{\abstractname}

%TODO: Abstract, check again for correctness
Virtual reality (\gls{VR}) and modern low-level graphics APIs, such as Vulkan, are hot topics in the field of high performance real-time graphics. Especially enterprise VR applications show the need for fast and highly optimized rendering of complex industrial scenes with very high object counts. However, solutions often need to be custom-tailored and the use of middleware is not always an option. Optimizing a Vulkan graphics renderer for high performance VR applications is a significant task. This thesis researches and presents a number of suitable optimization approaches. The goal is to integrate them into an existing renderer intended for enterprise usage, benchmark the respective performance impact in detail and evaluate those results. This thesis likewise includes all research and development documentation of the project, an explanation of successes and failures during the project and finally an outlook on how the findings may be used further.  





\makeatletter
\ifthenelse{\pdf@strcmp{\languagename}{english}=0}
{\renewcommand{\abstractname}{Kurzfassung}}
{\renewcommand{\abstractname}{Abstract}}
\makeatother

\chapter{\abstractname}

%TODO: Abstract in other language
\begin{otherlanguage}{ngerman} % TODO: select other language, either ngerman or english !
Sowohl \gls{VR} als auch low-level Grafikschnittstellen wie Vulkan sind stark gefragte Themen im Bereich der Echtzeitgrafik. Die Darstellung komplexer und hoch detaillierter Szenen ist besonders im industriellen Umfeld für VR Applikationen eine Grundvoraussetzung. Oftmals werden allerdings spezifisch zugeschnittene Lösungen benötigt, wenn vorgefertigt verfügbare Tools keine Option sind. Daher ist die Optimierung eines Vulkan Grafikrenderers für High Performance VR Anwendungen eine Kernaufgabe. Diese Arbeit untersucht und präsentiert demnach eine Anzahl an passenden Optimierungswegen. Teil des Ziels ist es, eine Auswahl dieser in einen vorhandenen Renderer aus dem industriellen Umfeld zu integrieren, die jeweilige Performance zu im Detail zu messen und auszuwerten. Diese Arbeit beinhaltet ebenso jegliche Forschungs- und Entwicklungsdokumentation des praktischen Projekts und gibt eine Erläuterung der Erfolge und Misserfolge und einen Ausblick darauf, wie die Ergebnisse womöglich einsetzbar sind. 
\end{otherlanguage}


% Undo the name switch
\makeatletter
\ifthenelse{\pdf@strcmp{\languagename}{english}=0}
{\renewcommand{\abstractname}{Abstract}}
{\renewcommand{\abstractname}{Kurzfassung}}
\makeatother